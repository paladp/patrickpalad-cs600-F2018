\documentclass[11pt]{article}

\newcommand{\assignmentduedate}{October 2}
\newcommand{\assignmentassignedate}{September 18}
\newcommand{\assignmentnumber}{Two}

\newcommand{\labyear}{2018}
\newcommand{\labday}{Tuesday}
\newcommand{\labtime}{2:30 pm}

\newcommand{\assigneddate}{Assigned: \labday, \assignmentassignedate, \labyear{}}
\newcommand{\duedate}{Due: \labday, \assignmentduedate, \labyear{} at \labtime{}}

\usepackage{pifont}
\newcommand{\checkmark}{\ding{51}}
\newcommand{\naughtmark}{\ding{55}}

\usepackage{listings}
\lstset{
  basicstyle=\small\ttfamily,
  columns=flexible,
  breaklines=true
}

\usepackage{fancyhdr}

\usepackage[margin=1in]{geometry}
\usepackage{fancyhdr}

\pagestyle{fancy}

\fancyhf{}
\rhead{Computer Science 600}
\lhead{Project Assignment \assignmentnumber{}}
\rfoot{Page \thepage}
\lfoot{\duedate}

\usepackage{titlesec}
\titlespacing\section{0pt}{6pt plus 4pt minus 2pt}{4pt plus 2pt minus 2pt}

\newcommand{\projecttitle}[1]
{
  \begin{center}
    \begin{center}
      \bf
      CMPSC 600\\Senior Thesis I\\
      Fall 2018\\
      \medskip
    \end{center}
    \bf
    #1
  \end{center}
}

\usepackage{xspace}

%% Words that we need to spell consistently
\newcommand{\website}{web site\xspace}
\newcommand{\webpage}{web page\xspace}
\newcommand{\webpages}{web pages\xspace}
\newcommand{\websites}{web sites\xspace}
\newcommand{\stylesheet}{style sheet\xspace}
\newcommand{\stylesheets}{style sheets\xspace}
\newcommand{\Viewport}{Viewport\xspace}
\newcommand{\viewport}{viewport\xspace}
\newcommand{\viewports}{viewports\xspace}
\newcommand{\XPath}{XPath\xspace}
\newcommand{\xpath}{\XPath\xspace}
\newcommand{\etal}{et al.\xspace}

%% Numbers of things
\newcommand{\numquestions}{two\xspace}
\newcommand{\numtypes}{five\xspace}
\newcommand{\numalgorithms}{four\xspace}
\newcommand{\numtrials}{30\xspace}
\newcommand{\numsubjects}{25\xspace}
\newcommand{\maxnumhtmlelements}{1469\xspace}
\newcommand{\minnumhtmlelements}{41\xspace}
\newcommand{\totalfailures}{24\xspace}
\newcommand{\totalviewports}{192\xspace}
\newcommand{\maxpagefailures}{5\xspace}
\newcommand{\numpageswithfailures}{13\xspace}
\newcommand{\numpagesthatarefast}{15\xspace}
\newcommand{\numtruepositives}{192\xspace}
\newcommand{\numtwoormorefailures}{8\xspace}
\newcommand{\numstackoverflowquestions}{961,157\xspace}
\newcommand{\wminvalue}{400}
\newcommand{\wmaxvalue}{1400}
\newcommand{\wpairvalue}{(\wminvalue,\wmaxvalue)}

%% Tools
\newcommand{\redecheck}{{\sc ReDeCheck}\xspace}
\newcommand{\xpert}{{\sc X-Pert}}
\newcommand{\gwali}{{\sc gwali}}
\newcommand{\cornipickle}{{\sc Cornipickle}}
\newcommand{\wraith}{{\sc Wraith}}
\newcommand{\galen}{{\sc Galen}}
\newcommand{\fighting}{{\sc Fighting\,Layout\,Bugs}}
\newcommand{\responsivepx}{{\sc ResponsivePX}}
\newcommand{\redecheckexplained}{\redecheck~(REsponsive DEsign CHECKer, pronounced ``Ready Check'')\xspace}

\begin{document}

\thispagestyle{empty}

\projecttitle{Project \assignmentnumber{} \\ \assigneddate{} \\ \duedate{}}

% TODO: Delete this title and add one for your senior thesis
\section*{Impact of Economic Policy on Income Distribution: An Agent Based Approach}

% TODO: Delete this abstract and include your own thesis abstract
\section*{Senior Thesis Abstract}

Arguments surrounding the idea of a living wage and its legitimacy as an economic policy have been garnering
increasing visibility within the general public. The gap between socioeconomic classes is widening, and
real wages are declining, and solutions aimed at tackling poverty, unemployment, and production are constantly
considered in a post financial crisis United States. Stylized facts and intuitive theory exist that can help
inform policy decisions, but strong predictive models are lacking. Related fields of economics - specifically
finance - have found success in computational agent based simulations to aide in the development and support
of theories and models. This paper aims to extend the work of agent based simulation to observe the impact of
minimum wage and other economic policies on the income distributions in an economy. The model used will replicate
an economy using heterogeneous agents - representing households, firms, banks, etc. - whose actions with each other
and their environment will be based on existing economic research. Changes in the environment the agents operate in
will represent the different economic policy options. The results of the simulation will provide insights into how
different policies effect the characteristics of the agents in the simulation. In addition to providing insight into the
effects of different policies, it will also further the literature by adding additional features to existing
agent based models.


% TODO: List your accomplishments towards demonstrating feasibility
\section*{Demonstrations of Feasibility}

\begin{enumerate}
  \item First feasibility indicator: Related research
A paper from Haldane et al. titled "Drawing on different disciplines: macroeconomic agent-based models" details
the history of modeling in Macroeconomics, and discusses how agent based models are poised to tackle problems in
economics that traditional stochastic general equilibrium models are less suited to address. These problems are
those that require agent heterogeneity and involve markets that are not always in equilibrium, among others. I believe that
a paper researching the topic of income distribution meets enough of these requirements that using an agent based
approach would be feasible and yield useful results. Additionally, a paper from Caiani et al titled "Agent based-stock
flow consistent macroeconomics: Towards a benchmark model" provides a detailed review of the common issues relating to
agent based models, as well as providing a template model for researchers to use and extend to their particular question.
I intend to rely on this paper for the basis of the model implemented in the simulation, and extend it to simulate specific
agent actions and attributes relevant to a study of income distribution.
  \item Second feasibility indicator: Possible implementations
In order to implement the computational side of the research, I plan to first rely on NetLogo as a means of prototyping
basic interactions between various agents, as well as familiarizing myself with the fundamental concepts of agent based
systems. I plan on implementing the final program using Java. This is due to two reasons; first, Java is the language I am
most comfortable using, and secondly, there is extensive support for this type of simulation through other Java packages and
suites. For example Repast is a suite of Java packages for agent based modeling that has been under development for over a decade.
JADE (Java Agent Development) is another similar framework.
  \end{enumerate}
\end{document}
